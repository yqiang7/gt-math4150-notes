\chapter{Nov.~12 --- Diophantine Equations}

\section{Linear Diophantine Equations}

\begin{definition}
  Any equation with one or more variables
  to be solved in the integers is called a
  \emph{Diophantine equation}.
\end{definition}

\begin{example}
  We can consider $5x^2 - 2x + 1 = 0$
  as a Diophantine equation.
\end{example}

\begin{definition}
  Let $a_1, \dots, a_n, b \in \Z$
  with $a_1, \dots, a_n \ne 0$. A
  Diophantine equation of the form
  \[
    a_1 x_1 + a_2 x_2 + \dots + a_n x_n = b
  \]
  is called a \emph{linear Diophantine equation}.
\end{definition}

\begin{theorem}
  Let $ax = b$ be a linear Diophantine
  equation in the variable $x$.
  If $a \mid b$, then there is a
  unique solution $x = b / a$. If
  $a \nmid b$, then there is no solution.
\end{theorem}

\begin{theorem}
  Let $ax + by = c$ be a linear Diophantine
  equation in the variables $x, y$.
  Let $d = (a, b)$. If $d \nmid c$,
  then there are no solutions.
  If $d \mid c$, there are infinitely
  many solutions, given by
  \[
    x = x_0 + (b / d) n \quad \text{and} \quad
    y = y_0 - (a / d) n, \quad
    n \in \Z
  \] 
  for a particular solution
  $x_0, y_0 \in \Z$.
\end{theorem}

\begin{proof}
  If there were a solution
  $x, y \in \Z$, then
  $d \mid a$ and $d \mid b$, so
  then $d \mid ax + by = c$.
  Taking the contrapositive implies that
  if $d \nmid c$, then there cannot be
  any solutions.

  Now assume that $d \mid c$.
  Using the Euclidean algorithm we can find
  $r, s \in \Z$ such that
  \[
    d = (a, b) = ra + sb.
  \]
  Further, if $d \mid c$, then $c = dq$ 
  for some $q \in \Z$, so we may write
  \[
    c = (ra + sb) q
    = a(rq) + b(sq).
  \]
  Thus $x = rq$ and
  $y = sq$ is a particular solution.

  Now, let $x_0, y_0$ be any particular
  solution and $x = x_0 + (b / d) n$,
  $y = y_0 - (a / d) n$ for some
  $n \in \Z$. Then
  \begin{align*}
    ax + by
    &= a(x_0 + (b / d) n) + b(y_0 - (a / d) n) \\
    &= ax_0 + by_0 + nab / d - nab / d
    = ax_0 + by_0
    = c,
  \end{align*}
  so $x, y$ is a solution
  for any integer $n$.

  Finally, we check that every solution
  is of this form.
  Let $x, y$ be any solution. Note that
  \[
    0 = (ax + by) - (ax_0 + by_0)
    = a(x - x_0) + b(y - y_0),
  \]
  so $a(x - x_0) = b(y_0 - y)$.
  Dividing both sides by $d$, we get
  \[
    \frac{a}{d} (x - x_0)
    = \frac{b}{d} (y_0 - y).
  \] 
  Since $d = (a, b)$, we have
  $(a / d, b / d) = 1$, so
  $a / d \mid y_0 - y$. Thus
  $y_0 - y = (a / d) n$ for some $n \in \Z$,
  so we have
  $y = y_0 - (a / d) n$. Substituting
  this above, we get
  $x = x_0 + (b / d) n$.
\end{proof}

\begin{example}
  Determine if
  $803 x + 154 y = 11$ has solutions.
  If so, calculate all of them.

  Using the Euclidean algorithm, we
  can compute that
  \[
    (803, 154)
    = (33, 154)
    = (33, 22)
    = 11,
  \]
  where we used that
  \begin{align*}
    803 &= 5 \cdot 154 + 33 \\
    154 &= 4 \cdot 33 + 22 \\
    33 &= 1 \cdot 22 + 11.
  \end{align*}
  Thus we can write
  \begin{align*}
    11 &= 33 - 22
    = 33 - (154 - 4 \cdot 33) \\
       &= 5 \cdot 33 - 154
    = 5(803 - 5 \cdot 154) - 154 \\
       &= 5 \cdot 803 - 26 \cdot 154.
  \end{align*}
  Thus $22 = 803 \cdot 10 + 154 \cdot (-52)$,
  so $(10, -52) $ is a particular solution.
  The other solutions are given by
  \[
    x = 10 + \frac{154}{11} n
    = 10 + 14 n \quad \text{and} \quad
    y = -52 - \frac{803}{11} n
    = -52 - 73 n, \quad n \in \Z.
  \]
\end{example}

\section{Nonlinear Diophantine Equations}

\begin{definition}
  A Diophantine equation is
  \emph{nonlinear} if it is not linear.
\end{definition}

\begin{example}
  The Diophantine equations
  $ax^2 + bx = c$
  and $5x^3 - 2 = 7x^{420}$ are nonlinear.
\end{example}

\begin{remark}
  We can now use a method that shows
  some equations are \emph{not}
  solvable. The idea is the following:
  if a Diophantine equation has solutions,
  then the equation will also have a
  solution when viewed as a congruence
  modulo any modulus. Taking the
  contrapositive, if a particular
  congruence modulo some modulus
  modulus is not solvable, then
  neither is the original equation.
\end{remark}

\begin{example}
  Prove that $3x^2 + 2 = y^2$ is
  not solvable.

  Assume by way of contradiction
  that there is a solution. Consider
  the equation modulo $3$:
  \[
    y^2 \equiv 3x^2 + 2
    \equiv 2 \pmod{3}.
  \]
  This says that $2$ is a quadratic
  residue modulo $3$, which is
  false. Therefore, there are no solutions.
\end{example}

\begin{example}
  Prove that $7x^3 + 2 = y^3$
  has no solutions.

  Consider the equation modulo $7$:
  \[
    y^3 \equiv 7x^3 + 2
    \equiv 2 \pmod{7}.
  \]
  This is solvable if and only if
  $2$ is a cubic residue modulo $7$, but
  we can compute that
  \[
    \{0^3, 1^3, 2^3, 3^3, 4^3, 5^3, 6^3\}
    \equiv
    \{0, 1, 1, 6, 1, 6, 6\} \pmod{7},
  \]
  so the only nonzero cubic residues are
  $1, 6$. Thus the congruence modulo $4$
  has no solutions, hence
  the original equation also has
  no solutions.
\end{example}

\begin{example}
  Prove that $x^2 + y^2 + 1 = 4z$ 
  has no solutions.

  Take this equation modulo $4$, we get
  the following congruence:
  \[
    x^2 + y^2 \equiv 3 \pmod{4}.
  \]
  The only quadratic residues modulo $4$
  are $\{0, 1\}$, and none of
  $0 + 0$, $0 + 1$, or $1 + 1$
  is equal to $3$, so there are no
  solutions to this congruence.
\end{example}

\begin{exercise}
  Prove that $x^2 + 2y^2 + 3 = 8z$ has
  no solutions.

  Take this equation modulo $8$:
  \[
    x^2 + 2y^2 \equiv 5 \pmod{8}.
  \]
  Note that the only quadratic residues
  modulo $8$ are $0, 1, 4$.
  Taking the above equation modulo
  $2$ we see that $x^2$ must be odd,
  so $x^2 \equiv 1 \Pmod{8}$. Then we get
  \[
    2y^2 \equiv 4 \pmod{8},
  \]
  so after dividing by $2$
  we get $y^2 \equiv 2 \Pmod{4}$, which is
  not possible.
\end{exercise}
