\chapter{Nov.~19 --- Fermat Descent}

\section{Fermat's Last Theorem for \texorpdfstring{$n=4$}{n=4}}

\begin{remark}
  The idea is the following: One
  shows that a given Diohpantine
  equation has no solutions in
  positive solutions by assuming
  the existence of such a solution and
  then constructing another solution in
  positive integers having one
  component strictly smaller than that
  same component of the original
  solution. This process cannot
  be continued ad infinitum, since
  it is not possible to construct an
  infinite strictly decreasing
  sequence of positive integers.
  Thus we obtain a contradiction.
\end{remark}

\begin{theorem}
  The Diophantine equation
  $x^4 + y^4 = z^2$ has no solutions
  in non-zero integers $x, y, z$.
\end{theorem}

\begin{proof}
  Assume by way of contradiction that
  that $x^4 + y^4 = z^2$ has a solution
  $x_1, y_1, z_1$ in non-zero integers.
  Without loss of generality, we
  may assume that $x_1, y_1, z_1 > 0$
  and $(x_1, y_1) = 1$. We will show
  that there is another solution
  $x_2, y_2, z_2$ in
  positive integers such that $(x_2, y_2) = 1$
  and $0 < z_2 < z_1$. Now
   $x_1^2, y_1^2, z_1$ is a
   Pythagorean triple with
   $(x_1^2, y_1^2, z_1) = 1$, and
   without loss of generality we
   may assume $y_1^2$ is even
   (thus $y_1$ is even). Thus by
   Theorem \ref{thm:pythagorean-triples},
   there exist $m, n \in \Z$ such
   that $m > n > 0$, $(m, n) = 1$,
   exactly one of $m, n$ is even, and
   \[
     x_1^2 = m^2 - n^2, \quad
     y_1^2 = 2mn, \quad
     z_1 = m^2 + n^2.
   \]
   Now $x_1^2 = m^2 - n^2$ implies that
   $x_1^2 + n^2 = m^2$, so
   $x_1, n, m$ is also a Pythagorean
   triple $(x_1, n, m) = 1$ and
   $n$ even. So $m$ is odd, and
   by Theorem \ref{thm:pythagorean-triples},
   there exist $a, b \in \Z$ such that
   $a > b > 0$, $(a, b) = 1$, exactly
   one of $a, b$ is even, and
   \[
     x_1 = a^2 - b^2, \quad
     n = 2ab, \quad
     m = a^2 + b^2.
   \]
   We wish to prove that $m, a, b$
   are perfect squares. Since
   $y_1^2 = 2mn = m(2n)$, and
   $(m, 2n) = 1$, we have that $m$
   and $2n$ must be perfect squares.
   Since $2n$ is a perfect square,
   there exists an integer
   $c \in \Z$ such that $2n = 4c^2$,
   or equivalently, $n = 2c^2$.
   Now $n = 2ab$ implies $c^2 = ab$.
   Since $(a, b) = 1$, we have that
   $a$ and $b$ must be perfect squares.
   Thus $m, a, b$ are all perfect
   squares, and there exist
   $x_2, y_2, z_2 \in \Z_{> 0}$
   such that
   $m = z_2^2$, $a = x_2^2$, and
   $b = y_2^2$. Then $m = a^2 + b^2$
   implies that $z_2^2 = x_2^4 + y_2^4$.
   Also $(x_2, y_2) = 1$ since
   $(a, b) = 1$, and we have
   \[
     0 < z_2 \le z_2^2 = m
     \le m^2 < m^2 + n^2 = z_1.
   \]
   This argument can be iterated
   arbitrarily many times, which
   gives a contradiction.
\end{proof}

\begin{remark}
  Any non-zero solution to
  $x^4 + y^4 = z^4$ gives a nonzero
  solution to $x^4 + y^4 = z^2$ by
  taking $z' = z^2$, so
  this also shows that
  $x^4 + y^4 = z^4$ has no
  solutions.
\end{remark}
