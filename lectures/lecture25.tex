\chapter{Dec.~1 --- Sums of Squares}

\section{Representations by Sums of Two Squares}

\begin{remark}
  Which integers are representable as
  a sum of two (integer) squares? In
  other words, for which integers
  $n$ are there solutions to the
  Diophantine equation
  $n = x^2 + y^2$?
\end{remark}

\begin{prop}\label{prop:square-sum-product}
  Let $n_1, n_2 \in \Z$ with
  $n_1, n_2 > 0$. If
  $n_1, n_2$ are both expressible as
  sums of two squares, then so is their
  product $n_1 n_2$.
\end{prop}

\begin{proof}
  Let $x_1, x_2, y_1, y_2 \in \Z$ 
  such that $n_1 = x_1^2 + y_1^2$ and
  $n_2 = x_2^2 + y_2^2$. Then
  \[
    n_1 n_2
    = (x_1^2 + y_1^2)(x_2^2 + y_2^2)
    = (x_1 x_2 + y_1 y_2)^2
    + (x_1 y_2 - x_2 y_1)^2,
  \]
  so $n_1 n_2$ is also expressible
  as a sum of two squares.
\end{proof}

\begin{lemma}\label{lem:square-sum-multiple}
  If $p$ is a prime and
  $p \equiv 1 \Pmod{4}$, then there exist
  $x, y \in \Z$ such that
  \[
    x^2 + y^2 = kp
  \]
  for some integer $0 < k < p$.
\end{lemma}

\begin{proof}
  Since $p \equiv 1 \Pmod{4}$, then
  $\big(\frac{-1}{p}\big) = 1$. Thus
  there exists $x \in \Z$ such that
  $0 < x \le (p - 1) / 2$ and
  \[
    x^2 + 1 \equiv 0 \pmod{p}.
  \]
  That is $x^2 + 1 = kp$ for some
  $k > 0$. Since
  $0 < x \le (p - 1) / 2$, we have
  \[
    kp = x^2 + 1
    \le \left(\frac{p - 1}{2}\right)^2 + 1
    < \left(\frac{p}{2}\right)^2 + 1
    < p^2.
  \]
  Dividing by $p$, we get that
  $k < p$. Clearly $k$ is positive,
  so we get the result.
\end{proof}

\begin{prop}\label{prop:prime-sum-two-squares}
  If $p$ is a prime with
  $p \not\equiv 3 \Pmod{4}$, then
  $p$ is a sum of two squares.
\end{prop}

\begin{proof}
  The statement is clear if $p = 2$,
  so we may assume that
  $p \equiv 1 \Pmod{4}$. Let
  $m$ be the least positive integer
  such that there exist
  $x, y \in \Z$ with
  $x^2 + y^2 = mp$. We will show that
  $m = 1$.

  Suppose to the contrary that
  $m \ge 2$. Then let $a, b \in \Z$ such
  that
  \[
    x \equiv a \Pmod{m}, \quad
    y \equiv b \Pmod{m}, \quad
    - m / 2 < a, b \le m / 2.
  \]
  Then $a^2 + b^2 \equiv x^2 + y^2 \equiv mp = 0 \Pmod{m}$.
  Thus $a^2 + b^2 = m k$ for some $k$.
  Note that
  \[
    (a^2 + b^2)(x^2 + y^2)
    = (m k)(m p)
    = m^2 kp.
  \]
  By the proof of Proposition
  \ref{prop:square-sum-product}, we can
  write
  $(a^2 + b^2)(x^2 + y^2) = (ax + by)^2 + (ay - bx)^2$.
  So
  \[
    (ax + by)^2 + (ay - bx)^2
    = m^2 k p. \tag{$*$}
  \]
  Note that since
  $a \equiv x \Pmod{m}$ and
  $b \equiv y \Pmod{m}$, we have
  \[
    ax + by \equiv x^2 + y^2 \equiv 0 \Pmod{m}
    \quad \text{and} \quad
    ay - bx \equiv xy - yx \equiv 0 \Pmod{m}.
  \]
  Thus $(ax + by) / m, (ay - bx) / m \in \Z$,
  and dividing by $m^2$ in $(*)$ gives
  \[
    \left(\frac{ax + by}{m}\right)^2
    + \left(\frac{ay - bx}{m}\right)^2
    = kp.
  \]
  Now it suffices to show that
  $k < m$. Since
  $-m / 2 < a, b \le m / 2$, we have
  \[
    km = a^2 + b^2
    \le \left(\frac{m}{2}\right)^2
    + \left(\frac{m}{2}\right)^2
    = \frac{m^2}{2},
  \]
  so $k \le m / 2 < m$. This
  contradicts the minimality of $m$.
\end{proof}

\begin{theorem}
  Let $n \in \Z$ with $n > 0$. Then
  $n$ is a sum of two squares if
  and only if every prime factor
  $p$ of $n$ with
  $p \equiv 3 \Pmod{4}$ occurs
  with even exponent in the
  prime factorization of $n$.
\end{theorem}

\begin{proof}
  $(\Rightarrow)$ Assume that
  $p^{2i + 1} \mid \mid n$
  (where $a \mid \mid b$ means that
  $a$ exactly divides $b$). Then we
  show that $p \equiv 1 \Pmod{4}$.
  Since $n$ is a sum of two squares,
  there exist $x, y \in \Z$
  such that $n = x^2 + y^2$.
  Let $d = (x, y)$ and
  $a = x / d$, $b = y / d$, so
  that $(a, b) = 1$. Also
  let $m = n / d^2$. Then
  \[
    a^2 + b^2 = m.
  \]
  Note that $(a, m) = 1$
  also, since if $p \mid a, m$ then
  $p \mid b$ as well, contradicting
  $(a, b) = 1$. The same argument
  implies $(b, m) = 1$ as well.
  Let $p^j \mid \mid d$.
  Then $p^{2(i - j) + 1} \mid \mid m$.
  Since $2(i - j) + 1 \ge 1$, we have
  $p \mid m$. But $p \nmid a$
  since $(a, m) = 1$, so
  the following congruence:
  \[
    az \equiv b \pmod{p}
  \]
  has a solution $z$, namely
  $z \equiv b \overline{a} \Pmod{p}$. Thus
  \[
    0 \equiv
    m \equiv a^2 + b^2
    \equiv a^2 + (az)^2
    \equiv a^2(1 + z^2) \pmod{p}.
  \]
  Thus $p \mid a^2 (1 + z^2)$, so
  $p \mid 1 + z^2$ as $p \nmid a$. So
  $-1$ is a quadratic residue
  modulo $p$,
  so $p \not\equiv 3 \Pmod{4}$.

  $(\Leftarrow)$ Suppose now that
  every prime factor $p$ of $n$
  with $p \equiv 3 \Pmod{4}$
  occurs to an even exponent. Then
  \[
    n = T^2 p_1 p_2 \cdots p_r
  \]
  for some distinct primes
  $p_1, \dots, p_r \not\equiv 3 \Pmod{4}$.
  Each $p_i$ is then a sum of two squares
  by Proposition \ref{prop:prime-sum-two-squares},
  hence so is their product
  $p_1 \cdots p_r$ by
  Proposition \ref{prop:square-sum-product}.
  Thus we can write $p_1 \cdots p_r = x^2 + y^2$,
  then $n = T^2 p_1 \cdots p_r = (Tx)^2 + (Ty)^2$
  is also a sum of two squares.
\end{proof}

\section{Representations by Sums of Three Squares}

\begin{remark}
  What about sums of 3 squares?
\end{remark}

\begin{prop}
  Let $m, n \in \Z$, $m, n \ge 0$.
  If $N = 4^m(8n + 7)$, then
  $N$ is not a sum of three squares.
\end{prop}

\begin{proof}
  Suppose first that $m = 0$, and
  assume to the contrary that
  $8n + 7$ is a sum of three squares.
  So there exist $x, y, z \in \Z$ such that
  \[
    8n + 7 = x^2 + y^2 + z^2.
  \]
  Note that if $a$ is even, then
  $a^2 \equiv 0, 4 \Pmod{8}$, and
  if $a$ is odd, then $a^2 \equiv 1 \Pmod{8}$.
  Then one explicitly checks that
  it is not possible to make a
  residue of $7$ modulo $8$ by
  summing $3$ of $\{0, 1, 4\}$, a
  contradiction.

  Suppose now that $N = 4^m(8n + 7)$ for
  general $m$ and assume
  $N = x^2 + y^2 + z^2$.
  Then
  $N \equiv 0, 4 \Pmod{8}$, and
  one checks (similarly as above) that
  none of $x, y, z$ can be odd.
  Thus $x = 2x'$, $y = 2y'$, $z = 2z'$, and
  \[
    4^m (8n + 7)
    = 4((x')^2 + (y')^2 + (z')^2),
  \]
  so $4^{m - 1} (8n + 7)$ is a sum of three squares.
  Repeat the argument until $m = 0$ to
  get a contradiction.
\end{proof}

\begin{remark}
  For numbers $n$ expressible as a sum of
  3 squares, one gets sets of numbers
  $(x, y, z)$ lying on the sphere
  $x^2 + y^2 + z^2 = n$. As $n$ grows,
  one can ask how the points $(x, y, z)$
  distribute on the sphere.
  These points are equidistributed
  as $n \to \infty$
  by a result of Duke.
\end{remark}

\section{Representations by Sums of Four Squares}

\begin{remark}
  What about sums of 4 squares?
\end{remark}

\begin{theorem}[Lagrange's four square theorem]
  Every integer
  $n \ge 0$ is a sum of four squares.
\end{theorem}

\begin{remark}
  His proof requires
  \emph{Euler's identity}: If
  $n_1 = x_1^2 + y_1^2 + z_1^2 + w_1^2$
  and $n_2 = x_2^2 + y_2^2 + z_2^2 + w_2^2$,
  then
  \begin{align*}
    n_1 n_2 
    &= (x_1 x_2 + y_1 y_2 + z_1 z_2 + w_1 w_2)^2
    + (-w_1 x_2 + x_1 w_2 - y_1 z_2 + y_2 z_1)^2 \\
    &\quad \quad + (-w_1 y_2 + y_1 w_2 - x_1 z_2 + x_2 z_1)^2
    + (-w_1 z_2 + z_1 w_2 - x_1 y_2 + x_2 y_1)^2.
  \end{align*}
  This is analogous to
  Proposition \ref{prop:square-sum-product}.
\end{remark}
