\chapter{Nov.~24 --- RSA Cryptosystem}

\section{Basics of Cryptography}
\begin{remark}
  The objective of \emph{cryptography}
  is to render communication
  unintelligible to all persons
  except the sender and the intended
  recipients. Today we will discuss
  \emph{public-key cryptography}.
\end{remark}

\begin{definition}
  The information to be encoded
  is called \emph{plaintext}, and
  the encoded message is called
  \emph{ciphertext}. The processes
  of going to ciphertext from plaintext
  is called \emph{encryption}, and
  the reverse is called \emph{decryption}.
\end{definition}

\begin{remark}
  We will convert all plaintext to
  numeric values first, e.g.
  \[
    \text{A} \to 01, \quad \text{B} \to 02, \quad
    \dots, \quad
    \text{Z} \to 26, \quad
    \text{\textvisiblespace} \to 27
  \]
\end{remark}

\begin{remark}
  The idea of \emph{public-key cryptography}
  is as follows: Any member of a
  network can send an encrypted
  message to any other member. This
  is done by associating a key to
  each individual that can be
  looked up and used: If these
  members are labeled
  $1, \dots, n$, then we have
  \begin{enumerate}
    \item A directory of public keys
      $e_1, \dots, e_n$.
    \item If a member of the
      network wants to encode and
      send a message to member $j$,
      then the key $e_j$ is used
      to encode the message.
    \item Each member has a decryption
      key $d_j$ known only to them,
      which is used to decrypt
      the messages sent to them back
      to plaintext.
    \item In principle, $d_j$ can
      be calculated from $e_j$, but it
      is sufficient if this cannot be
      done in a reasonable amount of
      time.
  \end{enumerate}
  The idea that underlies the dichotomy
  between $e_j$ and $d_j$ is that there
  is some mathematical operation
  which is ``cheap'' to perform in
  one direction but extremely
  ``expensive'' in the reverse
  direction. In RSA, this operation
  is the multiplication of two
  large primes (factoring is
  difficult).
\end{remark}

\section{RSA Cryptosystem}

\begin{remark}
  The \emph{RSA encryption algorithm}
  is a public-key cryptosystem
  introduced in 1977 by
  Ronald Rivest, Adi Shamir,
  and Len Adleman.
\end{remark}

\begin{definition}[RSA encryption scheme]
  The \emph{RSA encryption scheme}
  involves the following:
  \begin{enumerate}
    \item \emph{Public encryption keys:}
      Let $p, q$ be large primes
      (typically $\mathtt{\sim}1000$ digits each).
      Let $m = pq$, and $e$ a
      positive integer such that
      $(e, \varphi(m)) = 1$. Then the
      encryption key is the pair
      $(e, m)$.

      Note that $p, q, \varphi(m)$
      are \emph{not} made public.
    \item \emph{Formatting:} Translate
      each symbol into numeric values
      and arrange the resulting
      numerical string into blocks
      of length less than $m$.
      If a block is incomplete,
      pad it with ``dummy'' symbols.
    \item \emph{Encryption:}
      Given a block $P$ (think of
      $P$ as just a number
      $< m$), we encrypt via
      \[
        P^e \equiv C \pmod{m},
      \]
      where $C$ is viewed as a number
      $0 \le C < m$.
  \end{enumerate}
\end{definition}

\begin{example}\label{ex:rsa-encrypt}
  Consider the
  message ``I\textvisiblespace LIKE\textvisiblespace MATH'',
  which corresponds to the plaintext
  \[
    09 \quad 27 \quad 12 \quad 09
    \quad 11 \quad 05 \quad 27 \quad
    13 \quad 01 \quad 20 \quad 08.
  \]
  Grouping this into blocks of
  length $4$, we get:
  \[
    0927 \quad 1209 \quad 1105
    \quad 2713 \quad 0120 \quad
    0899.
  \]
  Taking $p = 53$, $q = 59$,
  $m = pq = 3127$,
  $\varphi(m) = 3016$,
  $e = 11$, we have
  \begin{align*}
    0927^{11} &\equiv 2982 \pmod{3127} \\
    1209^{11} &\equiv 1069 \pmod{3127} \\
    1105^{11} &\equiv 2619 \pmod{3127} \\
    2713^{11} &\equiv 2005 \pmod{3127} \\
    0120^{11} &\equiv 2579 \pmod{3127} \\
    0899^{11} &\equiv 0231 \pmod{3127}.
  \end{align*}
  Thus the ciphertext is
  \[
    2982\quad 1069\quad 2619\quad 2005\quad 2579\quad 0231.
  \]
\end{example}

\begin{definition}[RSA decryption scheme]
  The \emph{RSA decryption scheme}
  involves the following:
  \begin{enumerate}
    \item \emph{Private decryption key}:
      The decryption key is
      $(d, m)$, where
      $d$ is the inverse of $e$ modulo
      $\varphi(m)$.

      Note that calculating a modular
      inverse is \emph{not} an
      expensive operation
      (Euclidean algorithm),
      but this requires knowing
      $\varphi(m)$, which is not
      public.
    \item \emph{Decryption:}
      We decrypt a ciphertext block
      $C$ via
      \[
        C^d \equiv P \pmod{m}.
      \]
      Then concatenate the results
      and de-format in the obvious
      way.
  \end{enumerate}
\end{definition}

\begin{example}
  Consider the encrypted message
  $2982\ 1069\ 2619\ 2005\ 2579\ 0231$
  from Example \ref{ex:rsa-encrypt}.
  One can compute that $d = 1371$,
  so the decryption key is
  $(1371, 3127)$. The calculations are
  then
  \begin{align*}
    0927 &\equiv 2982^{1371} \pmod{3127} \\
    1209 &\equiv 1069^{1371} \pmod{3127} \\
    1105 &\equiv 2619^{1371} \pmod{3127} \\
    2713 &\equiv 2005^{1371} \pmod{3127} \\
    0120 &\equiv 2579^{1371} \pmod{3127} \\
    0899 &\equiv 0231^{1371} \pmod{3127},
  \end{align*}
  which recovers the original message
  after de-formatting.
\end{example}

\begin{remark}
  Why does this work? Suppose
  $0 \le P < m$ is some block
  to be encoded via
  $P^e \equiv C \Pmod{m}$.
  We have $ed \equiv 1 \Pmod{\varphi(m)}$.
  Thus $ed = 1 + k \varphi(m)$
  for some integer $k > 0$. Then
  if $(P, m) = 1$,
  \[
    C^d
    \equiv (P^e)^d
    \equiv P^{1 + k \varphi(m)}
    \equiv P (P^{\varphi(m)})^k
    \equiv P \pmod{m},
  \]
  where the last step is
  by Euler's theorem.
\end{remark}

\begin{exercise}
  Let $m, P > 1$,
  $(e, \varphi(m)) = 1$,
  and $ed \equiv 1 \Pmod{\varphi(m)}$.
  If $m$ is squarefree (but
  $m, P$ not necessarily
  coprime), show that
  \[
    P^{ed} \equiv P \pmod{m}.
  \]
  In particular, the RSA decryption
  step still works for
  $P$ and $m = pq$ with
  $(P, m) \ne 1$.
\end{exercise}
