\chapter{Nov.~10 --- Index Arithmetic}

\section{Index Arithmetic}

\begin{remark}
  Recall that if $r$ is a primitive
  root modulo $m$, then the set
  \[
    \{r, r^2, r^3, \dots, r^{\varphi(m)}\}
  \]
  is a reduced residue system modulo $m$.
\end{remark}

\begin{definition}
  Let $r$ be a primitive root modulo $m$.
  If $(a, m) = 1$, then the
  \emph{index of $a$ relative to $r$},
  denoted $\ind_r a$, is the least
  positive integer $n$ for which
  $r^n \equiv a \Pmod{m}$.
\end{definition}

\begin{remark}
  The index $\ind_r a$ always
  exists and satisfies
  $1 \le \ind_r a \le \varphi(m)$.
\end{remark}

\begin{example}
  Recall that $3$ is a primitive root
  modulo $7$. We can compute that
  \begin{align*}
    3^1 \equiv 3 \pmod{7}, \\
    3^2 \equiv 2 \pmod{7}, \\
    3^3 \equiv 6 \pmod{7}, \\
    3^4 \equiv 4 \pmod{7}, \\
    3^5 \equiv 5 \pmod{7}, \\
    3^6 \equiv 1 \pmod{7}.
  \end{align*}
  Thus we see that the indices are
  \begin{align*}
    \ind_3 3 = 1, \\
    \ind_3 2 = 2, \\
    \ind_3 6 = 3, \\
    \ind_3 4 = 4, \\
    \ind_3 5 = 5, \\
    \ind_3 1 = 6.
  \end{align*}
\end{example}

\begin{remark}
  If $a, b$ are coprime with $m$
  and $a \equiv b \Pmod{m}$, then
  $\ind_r a = \ind_r b$.
\end{remark}

\begin{remark}
  Indices enjoy similar
  properties as logarithms.
\end{remark}

\begin{prop}
  Let $r$ be a primitive root modulo $m$
  and $a, b \in \Z$ coprime to $m$. Then
  \begin{enumerate}
    \item $\ind_r 1 \equiv 0 \Pmod{\varphi(m)}$,
    \item $\ind_r r \equiv 1 \Pmod{\varphi(m)}$,
    \item $\ind_r(ab) \equiv \ind_r a + \ind_r b \Pmod{\varphi(m)}$,
    \item $\ind_r(a^n) \equiv n \ind_r a \Pmod{\varphi(m)}$ if $n > 0$
      is an integer.
  \end{enumerate}
\end{prop}

\begin{proof}
  (1)-(2) These are clear.

  (3) By definition, we have
  $r^{\ind_r a} \equiv a \Pmod{m}$
  and $r^{\ind_r b} \equiv b \Pmod{m}$.
  Thus
  \[
    r^{\ind_r a + \ind_r b}
    \equiv ab
    \equiv r^{\ind_r(ab)} \pmod{m}.
  \]
  By Proposition \ref{prop:order-congruence},
  we have
  $\ind_r a + \ind_r b \equiv \ind_r(ab) \Pmod{\varphi(m)}$, since
  $\ord_m r = \varphi(m)$.

  (4) We argue similarly as in (3):
  By definition,
  $r^{\ind_r a^n} \equiv a^n \Pmod{m}$.
  Also,
  \[
    r^{n \ind_r a}
    \equiv (r^{\ind_r a})^n
    \equiv a^n \pmod{m},
  \]
  so again by Proposition \ref{prop:order-congruence},
  $n \ind_r a \equiv \ind_r(a^n) \Pmod{\varphi(m)}$.
\end{proof}

\begin{example}
  We work modulo $7$ with primitive root
  $3$. Then $\ind_3 2 = 2$ and
  $\ind_3 3 = 1$, so
  \[
    \ind_3 6
    \equiv \ind_3 (2 \cdot 3)
    \equiv \ind_3 2 + \ind_3 3
    \equiv 3 \pmod{6}.
  \]
  So $\ind_3 6 = 3$, which agrees with
  our previous calculations.
\end{example}

\begin{remark}
  Suppose $r$ is a primitive
  root modulo $m$ and
  $(a, m) = (b, m) = 1$. Consider
  for $n > 0$
  \[
    a x^n \equiv b \pmod{m}.
  \]
  This congruence is equivalent to
  $\ind_r(b) \equiv \ind_r(ax^n) \equiv \ind_r(a) + n \ind_r(x) \Pmod{\varphi(m)}$,
  so
  \[
    n \ind_r(x) \equiv \ind_r(b) - \ind_r(a) \Pmod{\varphi(m)}.
  \]
  This is now a linear congruence which
  is equivalent to the original one.
\end{remark}

\begin{example}
  Use indices to find all incongruent
  solutions to
  \[
    6x^4 \equiv 3 \pmod{7}.
  \]
  Since $3$ is a primitive root, we can
  compute that
  \[
    4 \ind_3(x)
    \equiv \ind_3(3) - \ind_3(6)
    \equiv 1 - 3
    \equiv 4 \Pmod{6}.
  \]
  Note $(4, 6) = 2$ and
  $2 \mid 4$, so this congruence
  has $2$ solutions by
  Theorem \ref{thm:linear-solutions-number}. Dividing by $2$, we get
  \[
    2\ind_3(x) \equiv 2 \Pmod{3},
  \]
  so $\ind_3(x) \equiv 1 \Pmod{3}$.
  Thus $\ind_3(x) \equiv 1, 4 \Pmod{6}$.
  This corresponds to
  $x \equiv 3^1, 3^4 \equiv 3, 4 \Pmod{7}$.
\end{example}

\section{Power Residues}
\begin{definition}
  Let $a, m, n \in \Z$ with
  $m, n > 0$ and $(a, m) = 1$. Then
  $a$ is an \emph{$n$th power residue modulo $m$}
  if the congruence $x^n \equiv a \Pmod{m}$
  has a solution.
\end{definition}

\pagebreak

\begin{example}
  We have the following:
  \begin{enumerate}
    \item $6$ is a $3$rd power residue modulo $7$ (we have calculated $\ind_3 6 = 3$).
    \item $3$ is a $4$th power residue modulo $13$ (one has $2^4 \equiv 16 \equiv 3 \Pmod{13}$).
    \item $3$ is not a $4$th power residue modulo $7$ (exercise).
  \end{enumerate}
\end{example}

\begin{theorem}\label{thm:nth-power-residue-criterion}
  Let $a, m, n \in \Z$ with $m, n > 0$
  and $(a, m) = 1$. If $m$ has a primitive
  root, then $a$ is an $n$th power residue
  modulo $m$ if and only if
  \[
    a^{\varphi(m) / d} \equiv 1 \pmod{m},
  \]
  where $d = (n, \varphi(m))$. Furthermore,
  in this case, the congruence $x^n \equiv a \Pmod{m}$
  has exactly $d$ incongruent
  solutions modulo $m$.
\end{theorem}

\begin{proof}
  Let $r$ be a primitive root modulo $m$.
  Then the congruence
  $x^n \equiv a \Pmod{m}$ is equivalent to
  \[
    n \ind_r(x) \equiv \ind_r(a) \pmod{\varphi(m)}.
  \]
  This congruence is solvable if and only if
  $d = (n, \varphi(m))$ divides
  $\ind_r a$
  by Theorem \ref{thm:linear-solutions-number}.
  In this case, there are $d$ incongruent
  solutions. The condition
  $d \mid \ind_r a$ is equivalent to
  \[
    \frac{\varphi(m)}{d} \ind_r a
    \equiv 0 \pmod{\varphi(m)}.
  \]
  The forward implication is clear,
  and the congruence implies
  $(\varphi(m) / d) \ind_r a = k \varphi(m)$
  for some $k \in \Z$, so
  $\ind_r a = dk$, so $d \mid \ind_r a$.
  The congruence is equivalent to
  $a^{\varphi(m) / d} \equiv 1 \Pmod{m}$.
\end{proof}

\begin{corollary}[Euler's criterion]
  Let $p$ be an odd prime and $a \in \Z$ 
  with $p \nmid a$. Then $a$ is a
  quadratic residue modulo $p$ if
  and only if
  \[
    a^{(p - 1) / 2} \equiv 1 \pmod{p}.
  \]
  Moreover, if this is the case, there
  are exactly $2$ incongruent solutions
  to $x^2 \equiv a \Pmod{p}$.
\end{corollary}

\begin{proof}
  Take $m = p$ and $n = 2$ in Theorem
  \ref{thm:nth-power-residue-criterion}.
\end{proof}

\begin{example}
  Let $a = 6$, $m = 7$, $n = 3$.
  A primitive root modulo $7$ is $3$.
  The congruence
  \[
    x^3 \equiv 6 \pmod{7}
  \]
  has $d = (3, \varphi(7)) = (3, 6) = 3$
  solutions, so $6$ is a $3$rd power
  residue modulo $7$.
\end{example}

\begin{example}
  Find all $15$th power residues modulo $9$.
  Since $9$ has a primitive root by the
  primitive root theorem, the congruence
  $x^{15} \equiv a \Pmod{9}$ has
  has solutions if and only if
  \[
    a^{\varphi(9) / d}
    \equiv 1 \pmod{9},
  \]
  where $d = (15, \varphi(9)) = (15, 6) = 3$.
  Thus we must have
  $1 \equiv a^{6 / 3} \equiv a^2 \Pmod{9}$,
  so $a \equiv \pm 1 \Pmod{9}$
  are the only solutions. So the
  only $15$th power residues modulo $9$
  are $\pm 1$.
\end{example}
