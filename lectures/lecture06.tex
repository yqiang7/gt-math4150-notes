\chapter{Sept.~8 --- Congruences, Part 2}

\section{More on Congruences}
\begin{example}
  Compute a complete residue system
  modulo $5$ using
  \begin{itemize}
    \item only even numbers:
      $\{0, 2, 4, 6, 8\}$,
    \item only prime numbers:
      $\{2, 3, 5, 11, 19\}$.
  \end{itemize}
\end{example}

\begin{example}
  Compute a complete residue system
  modulo $5$ using only numbers
  $\equiv 1 \Pmod{4}$.
\end{example}

\begin{remark}
  Recall that the set of equivalence
  classes of $\Z$ modulo $m$ form a ring.
  In particular, we can construct
  addition and multiplication tables.
  For $m = 4$, this looks like:
  \begin{center}
    \begin{tabular}{c|cccc}
      + & 0 & 1 & 2 & 3 \\
      \hline
      0 & 0 & 1 & 2 & 3 \\
      1 & 1 & 2 & 3 & 0 \\
      2 & 2 & 3 & 0 & 1 \\
      3 & 3 & 0 & 1 & 2
    \end{tabular}
    \quad \quad \quad
    \begin{tabular}{c|cccc}
      $\times$ & 0 & 1 & 2 & 3 \\
      \hline
      0 & 0 & 0 & 0 & 0 \\
      1 & 0 & 1 & 2 & 3 \\
      2 & 0 & 2 & 0 & 2 \\
      3 & 0 & 3 & 2 & 1
    \end{tabular}
  \end{center}
  Addition modulo $5$ is similar, but
  the multiplication table for $m = 5$ is:
  \begin{center}
    \begin{tabular}{c|ccccc}
      $\times$ & 0 & 1 & 2 & 3 & 4 \\
      \hline
      0 & 0 & 0 & 0 & 0 & 0 \\
      1 & 0 & 1 & 2 & 3 & 4 \\
      2 & 0 & 2 & 4 & 1 & 3 \\
      3 & 0 & 3 & 1 & 4 & 2 \\
      4 & 0 & 4 & 3 & 2 & 1
    \end{tabular}
  \end{center}
  Recall that a ring with no zero divisors
  (nonzero elements $a, b$ such that $ab = 0$)
  is an \emph{integral domain}, in
  particular we see from the multiplication
  table that $\Z / 5\Z$ is an integral
  domain. Since a finite integral domain
  is automatically a \emph{field}, we
  see that $\Z / 5\Z$ is a field.
\end{remark}

\begin{prop}
  Let $a, b, c, m, \in \Z$ with $m > 0$.
  Then
  \[
    ca \equiv cb \Pmod{m}
    \quad \text{if and only if} \quad
    a \equiv b \Pmod{m / (m, c)}.
  \]
  In particular, if $m$ is prime, then
  $ca \equiv cb \Pmod{m}$
  if and only if $a \equiv b \Pmod{m}$
  for $c \not\equiv 0 \Pmod{m}$.
\end{prop}

\begin{proof}
  $(\Rightarrow)$ We have
  $ca \equiv cb \Pmod{m}$ if and only if
  $m \mid ca - cb = c(a - b)$. Let
  $d = (m, c)$. By the transitivity of
  divisibility, we have
  $(m / d) \mid (c / d)(a - b)$.
  But $(m / d, c / d) = 1$, so
  $(m / d) \mid a - b$. Then we
  have $a \equiv b \Pmod{m / d}$
  by the definition of congruence.

  $(\Leftarrow)$ Again let $d = (m, c)$.
  Then $a \equiv b \Pmod{m / d}$, so
  $(m / d) \mid a - b$. Then
  $m \mid d(a - b)$, and so
  \[
    m \mid d(a - b)(c / d)
    = c(a - b) = ca - cb,
  \]
  which means $ca \equiv cb \Pmod{m}$
  by the definition of congruence.
\end{proof}

\begin{remark}
  This shows that the congruence
  classes modulo $m$ form a field if
  and only if $m$ is prime.
\end{remark}

\section{Linear Congruences in One Variable}

\begin{definition}
  Let $a, b \in \Z$. A congruence of
  the form
  \[
    ax \equiv b \Pmod{m}
  \]
  is called a \emph{linear congruence} in
  the variable $x$.
\end{definition}

\begin{example}
  Consider the following linear congruences:
  \begin{itemize}
    \item $2x \equiv 3 \Pmod{4}$
      has no solutions;
    \item $2x \equiv 4 \Pmod{6}$
      has $x = 2, 5$ as solutions;
    \item $3x \equiv 9 \Pmod{6}$ has
      $x = 1, 3, 5$ as solutions.
  \end{itemize}
\end{example}

\begin{theorem}\label{thm:linear-solutions-number}
  Let $ax \equiv b \Pmod{m}$, and
  let $d = (a, m)$. If $d \nmid b$, then
  there are no solutions for $x$ in $\Z$. If
  $d \mid b$, then the congruence
  has exactly $d$ incongruent solutions
  modulo $m$ in $\Z$.
\end{theorem}

\begin{proof}
  Note that $ax \equiv b \Pmod{m}$ if
  and only if $m \mid ax - b$, if
  and only if $ax - b = my$ for some
  integer $y$. This is equivalent to
  $ax - my = b$. Thus $ax \equiv b \Pmod{m}$
  is solvable in $x$ if and only if
  the equation $ax - my = b$ is solvable
  in $x, y$.

  Let $x, y$ be a solution of
  $ax - my = b$. Since $d \mid a$ and
  $d \mid m$, we must have $d \mid b$.
  Taking contrapositives, this
  proves the first part of the theorem.

  Assume now that $d \mid b$. We prove
  the second part in 4 steps:
  \begin{enumerate}
    \item We will show that $ax \equiv b \Pmod{m}$
      has a solution $x_0$.
    \item We will show that there are
      infinitely many solutions
      of a particular form involving $x_0$.
    \item We will show that
      any solution has a particular
      form involving $x_0$. (Note that
      this combines with $(2)$ to give
      all possible solutions.)
    \item We will show that there are
      exactly $d$ equivalence classes
      of solutions.
  \end{enumerate}

  $(1)$ Since $d = (a, m)$, there exist
  $r, s \in \Z$ such that $d = ra + sm$.
  Since $d \mid b$, we can write
  \[
    b = \frac{b}{d} \cdot d
    = \frac{b}{d}(ra + sm)
    = \frac{br}{d} \cdot a + \frac{bs}{d} \cdot m.
  \]
  Thus $b - a(b r/ d) = (b s/ d) m$,
  so $m \mid b - a(b r/ d)$, so
  $a(br / d) \equiv b \Pmod{m}$.
  Thus $x_0 = br / d$ is a solution.

  $(2)$ Let $x_0$ be any solution of
  $ax \equiv b \Pmod{m}$. Consider
  $x_0 + (m / d) n$ for $n \in \Z$. Then
  \[
    a (x_0 + (m / d) n)
    \equiv a x_0 + a (m / d) n
    \equiv b + (a / d)mn
    \equiv b \pmod{m},
  \]
  so $x_0 + (m / d) n$ is also solution
  for any $n \in \Z$.

  $(3)$ Let $x_0$ be a solution of
  $ax \equiv b \Pmod{m}$. Recall from the
  beginning of the proof that this is
  equivalent to there being
  $y_0 \in \Z$ such that
  $ax_0 - my_0 = b$. Let $x$ be any other
  solution. Then $ax - my = b$
  for some $y \in \Z$, so
  \[
    0 = b - b = (ax_0 - my_0) - (ax - my)
    = a(x_0 - x) - m(y_0 - y),
  \]
  which gives $a(x_0 - x) = m(y_0 - y)$.
  This is equivalent to
  $(a / d)(x_0 - x) = (m / d)(y_0 - y)$.
  Note that if $y_0 - y = 0$, then
  $x_0 - x = 0$ as well since
  $a / d \ne 0$. So we may assume
  $y_0 - y \ne 0$. Then
  \[(m / d) \mid (a / d)(x_0 - x),\]
  and since $(a / d, m / d) = 1$, we have
  $(m / d) \mid (x_0 - x)$. Thus
  $x \equiv x_0 \Pmod{m / d}$. In
  particular, all solutions to
  $ax \equiv b \Pmod{m}$ are given by
  $x = x_0 + (m / d) n$ for
  $n \in \Z$ and any particular
  solution $x_0$.

  $(4)$ Let $x_0 + (m / d) n_1$ and
  $x_0 + (m / d) n_2$ be solutions.
  Then we have
  \[
    x_0 + (m / d) n_1
    \equiv x_0 + (m / d) n_2 \Pmod{m}
  \]
  if and only if
  $(m / d) n_1 \equiv (m / d) n_2 \Pmod{m}$.
  This happens if and only if
  $m \mid (m / d)(n_1 - n_2)$, if and only if
  $(m / d) (n_1 - n_2) = km$ for some
  $K \in \Z$, if and only if
  $n_1 - n_2 = kd$. In particular, this
  is equivalent to $n_1 \equiv n_2 \Pmod{d}$.
  Since there are exactly $d$ congruence
  classes for $n$, there are exactly $d$
  congruence classes of solutions as well,
  which completes the proof.
\end{proof}
