\chapter{Nov.~17 --- Pythagorean Triples}

\section{Classification of Pythagorean Triples}

\begin{definition}
  A triple $x, y, z$ of positive integers
  satisfying the Diophantine
  $x^2 + y^2 = z^2$ is said to be a
  \emph{Pythagorean triple}.
\end{definition}

\begin{example}
  The following are Pythagorean triples:
  $3, 4, 5$ and $5, 12, 13$.
\end{example}

\begin{remark}
  The triples $-3, 4, 5$ and
  $0, 1, 1$ are solutions to
  $x^2 + y^2 = z^2$,
  but they are not Pythagorean triples
  (as they are not positive integers).
\end{remark}

\begin{remark}
  We take the following conventions:
  \begin{itemize}
    \item We only describe solutions with
      $x, y, z > 0$.
    \item If $x, y, z$ is a Pythagorean
      triple and $(x, y, z) = d$, then
      $x / d, y / d, z / d$ is also a
      Pythagorean triple, and
      $(x / d, y / d, z / d) = 1$.
      Thus we will only describe
      Pythagorean triples $x, y, z$ with
      $(x, y, z) = 1$.
      Such triples are called \emph{primitive}.
    \item Under the assumption that
      $x, y, z$ is a primitive
      Pythagorean triple, we show
      that exactly one of $x, y$ is even.
      We can see this as follows:

      First assume that $x$ and $y$
      are both even. Then $2 \mid x$ and
      $2 \mid y$, and since
      $z^2 = x^2 + y^2$, we have
      $2 \mid z^2$, and so $2 \mid z$.
      This contradicts $(x, y, z) = 1$.

      Now assume that $x$ and $y$ are
      both odd. Then $z$ is even, so
      $z^2 \equiv 0 \Pmod{4}$, and
      $x^2 \equiv y^2 \equiv 1 \Pmod{4}$.
      Since $x^2 + y^2 = z^2$, we have
      that $1 + 1 \equiv 0 \Pmod{4}$,
      a contradiction.

      Thus exactly one of $x$ or $y$
      is even, so without loss of generality
      we will only describe Pythagorean
      triples where $y$ is even.
  \end{itemize}
\end{remark}

\begin{theorem}[Euclid]\label{thm:pythagorean-triples}
  There are infinitely many primitive
  Pythagorean triples $x, y, z$ with
  $y$ even. They are given by
  $x = m^2 - n^2$, $y = 2mn$,
  $z = m^2 + n^2$, where
  $m, n \in \Z$, $m > n > 0$,
  $(m, n) = 1$, and exactly one of
  $m$ or $n$ is even.
\end{theorem}

\begin{example}
  The case
  $m = 2$, $n = 1$ yields
  $3, 4, 5$, and the case
  $m = 3$, $n = 2$ yields $5, 12, 13$.
\end{example}

\begin{proof}[Proof of Theorem \ref{thm:pythagorean-triples}]
  We first show that given a primitive
  Pythagorean triple with $y$ even,
  there exist $m, n \in \Z$ with
  the properties described in the theorem.
  Since $y$ is even, $x$ and $z$ are both
  odd (see the argument from before).
  A similar argument shows that
  $(x, y) = (y, z) = (x, z) = 1$. Now
  \[
    y^2 = z^2 - x^2
    = (z + x)(z - x)
  \]
  Since $y, z + x, z - x$ are all even,
  dividing by $2$ everywhere gives
  \[
    \left(\frac{y}{2}\right)^2
    = \left(\frac{z + x}{2}\right)
    \left(\frac{z - x}{2}\right). \tag{$1$}
  \]
  Now we claim
  $((z + x) / 2, (z - x) / 2) = 1$.
  To see this, let
  $d = ((z + x) / 2, (z - x) / 2)$.
  Then $d \mid (z + x) / 2$ and
  $d \mid (z - x) / 2$, so taking linear
  combinations gives
  \begin{align*}
    d \mid \frac{z + x}{2} + \frac{z - x}{2} = z, \\
    d \mid \frac{z + x}{2} - \frac{z - x}{2} = x
  \end{align*}
  so $d = 1$ since $(x, z) = 1$.
  Thus $(1)$ implies that
  $(z + x) / 2$ and
  $(z - x) / 2$ are both perfect squares
  by the fundamental theorem of arithmetic.
  Let $m, n \in \Z$ positive such that
  \[
    \frac{z + x}{2} = m^2
    \quad\text{and}\quad
    \frac{z - x}{2} = n^2.
  \]
  Then $m > n > 0$, $(m, n) = 1$ since
  their squares are coprime, and
  \[
    m^2 - n^2 = x, \quad
    2mn = y, \quad m^2 + n^2 = z.
  \]
  Also, $(m, n) = 1$ implies
  that not both $m$ and $n$ are even.
  If $m$ and $n$ are both odd, then we have
  that
  \[
    z = m^2 + n^2 \quad \text{and} \quad
    x = m^2 - n^2
  \]
  are both even, contradicting the
  fact that $(x, z) = 1$. This completes
  the first part.

  The second part of the proof is to show
  that given $m$ and $n$ as described and
  \[
    x = m^2 - n^2, \quad
    y = 2mn, \quad
    z = m^2 + n^2,
  \]
  then $x, y, z$ is a primitive
  Pythagorean triple with $y$ even. We
  check that
  \[
    x^2 + y^2
    = (m^4 - 2m^2 n^2 + n^4)
    + 4m^2 n^2
    = m^4 + 2m^2 n^2 + n^4
    = (m^2 + n^2)^2 = z^2,
  \]
  so $x, y, z$ is indeed
  a Pythagorean triple. Clearly $y$ is
  even. It remains to prove
  that $(x, y, z) = 1$. To see this, let
  $(x, y, z) = d$. Since exactly
  one of $m$ or $n$ is even, we have
  that $x$ and $z$ are both odd. Then
  $d$ is odd, and so $d = 1$ or $p \mid d$
  for some odd prime $p$. Assume
  that $p \mid d$. Then $p \mid x$ and
  $p \mid z$, and so
  $p \mid z + x$ and $p \mid z - x$.
  Thus $p \mid m^2 + n^2 + m^2 - n^2 = 2m^2$
  and $p \mid m^2 + n^2 - (m^2 - n^2) = 2n^2$.
  Since $p$ is odd, we have $p \mid m^2$
  and $p \mid n^2$, from which
  $p \mid m$ and $p \mid n$. Then
  $(m, n) \ne 1$, a contradiction.
  Thus the triple $x, y, z$ is primitive,
  which completes the proof.
\end{proof}

\begin{remark}
  Given Pythagorean triples, it
  is natural to consider the equation
  $x^n + y^n = z^n$ for $n \ge 3$.
\end{remark}

\begin{theorem}[Wiles-Taylor, 1994]
  The Diophantine equation
  $x^n + y^n = z^n$ has no solutions
  in the non-zero integers  $x, y, z$
  for any integer $n \ge 3$.
\end{theorem}

\begin{remark}
  The proof for the general case is
  extremely difficult, but special
  cases are much easier. For example,
  the case $n = 4$ can be shown via
  \emph{Fermat descent}.
\end{remark}

\begin{exercise}
  Find all solutions in positive
  integers to $x^2 + 2y^2 = z^2$.
\end{exercise}
